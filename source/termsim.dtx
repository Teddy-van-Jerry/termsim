% \iffalse meta-comment
% !TEX program  = XeLaTeX
%<*internal>
\iffalse
%</internal>
%<*readme>
Terminal simulation based on tcolorbox and minted/listings
=======

`termsim` is a LaTeX package developed based on tcolorbox and minted/listings, which provides a environment `terminal`(`terminal*`) and a macro `\termfile`(`\termfile*`) to simulate Win10, Ubuntu and Mac terminal.

The `terminal`(`terminal*`) evironment create a tcolorbox to simulate terminal with the eviroment body and the `\termfile`(`\termfile*`) macro from a source file.

The OS type was selected from win10, mac, or ubuntu and
the terminal theme was selected from dark, light, or with key-value option.

All options can be set with `\termset` macro or `terminal`(`terminal*`) eniroment's and
`\termfile`(`\termfile*`) macro's option

You can read the manual (in Chinese) for more details and examples.

Contributing
------------

1. github repository:
    1. repository: [termsim](https://github.com/registor/termsim)
    2. Issues and pull requests are welcome. [issue](https://github.com/registor/termsim/issues) or [pull request](https://github.com/registor/termsim/pulls).

2. gitee repository:
    1. repository: [termsim](https://gitee.com/nwafu_nan/termsim)
    2. Issues and pull requests are welcome. [issue](https://gitee.com/nwafu_nan/termsim/issues) or [pull request](https://gitee.com/nwafu_nan/termsim/pulls).

基于tcolorbox的终端模拟宏包
=======

`termsim`是一个基于tcolorbox的用LaTeX3开发的终端模拟LaTeX宏包,它提供了用于模拟终端窗口的`terminal`(`terminal*`)环境和`\termfile`(`\termfile*`)命令,以模拟Win10、Ubuntu或Mac的终端窗口。

`terminal`(`terminal*`)环境根据环境内容创建一个`tcolorbox`盒子用于模拟终端窗口,
`\termfile`(`\termfile*`)命令根据载入的文件内容创建一个`tcolorbox`盒子用于模拟终端窗口。

可以通过`terminal`(`terminal*`)环境或`\termfile`(`\termfile*`)命令的`key-value`选项设置操作系统类型(win10、mac或ubuntu),并可选择暗色(dark)、亮色(light)或白色(white)颜色主题。

所有选项即可以通过环境或命令选项设置,也可以通过`\termset`命令进行设置。

可以通过阅读宏包手册(中文)以也解该宏包更多的使用细节和使用样例。

###  参与贡献
---------------------

1. github仓库:
    (1). 仓库地址: [termsim](https://github.com/registor/termsim)
    (2). Issues和PR: [issue](https://github.com/registor/termsim/issues) or [pull request](https://github.com/registor/termsim/pulls).

2. gitee仓库:
    (1). 仓库地址: [termsim](https://gitee.com/nwafu_nan/termsim)
    (2). Issues and PR: [issue](https://gitee.com/nwafu_nan/termsim/issues) or [pull request](https://gitee.com/nwafu_nan/termsim/pulls).

Copyright and Licence
---------------------

    Copyright (C) 2020-2021 by Nan Geng <nangeng@nwafu.edu.cn>
    ----------------------------------------------------------------------

    This work may be distributed and/or modified under the
    conditions of the LaTeX Project Public License, either
    version 1.3c of this license or (at your option) any later
    version. This version of this license is in
       http://www.latex-project.org/lppl/lppl-1-3c.txt
    and the latest version of this license is in
       http://www.latex-project.org/lppl.txt
    and version 1.3 or later is part of all distributions of
    LaTeX version 2005/12/01 or later.

    This work has the LPPL maintenance status "maintained".

    The Current Maintainer of this work is Nan Geng.

    This package consists of the file  termsim.dtx,
                 and the derived files termsim.sty,
                                       termsim.pdf,
                                       termsim.ins,
                                       README.md (this file).

%</readme>
%<*internal>
\fi
\begingroup
  \def\temp{LaTeX2e}
\expandafter\endgroup\ifx\temp\fmtname\else
\csname fi\endcsname
%</internal>
%<*install>

\input ctxdocstrip %

\let\MetaPrefix\relax

\preamble

    Copyright (C) 2020-2021 by Nan Geng <nangeng@nwafu.edu.cn>
--------------------------------------------------------------------------

    This work may be distributed and/or modified under the
    conditions of the LaTeX Project Public License, either
    version 1.3c of this license or (at your option) any later
    version. This version of this license is in
       http://www.latex-project.org/lppl/lppl-1-3c.txt
    and the latest version of this license is in
       http://www.latex-project.org/lppl.txt
    and version 1.3 or later is part of all distributions of
    LaTeX version 2005/12/01 or later.

    This work has the LPPL maintenance status "maintained".

    The Current Maintainer of this work is Nan Geng.

--------------------------------------------------------------------------

\endpreamble

\postamble

    This package consists of the file  termsim.dtx,
                 and the derived files termsim.sty,
                                       termsim.pdf,
                                       termsim.ins,
                                       README.md.
\endpostamble

\declarepostamble\emptypostamble
\endpostamble

\def\MetaPrefix{-- }


\let\MetaPrefix\DoubleperCent

\generate
  {
%</install>
%<*internal>
    \usedir{source/xelatex/termsim}
    \file{termsim.ins} {\from{\jobname.dtx}{install}}
%</internal>
%<*install>
    \usedir{xetex/xelatex/termsim}
    \file{termsim.sty} {\from{\jobname.dtx}{package}}
    \nopreamble\nopostamble
    \usedir{doc/xelatex/termsim}
    \file{README.md}   {\from{\jobname.dtx}{readme}}
  }

\endbatchfile
%</install>
%<*internal>
\fi
%</internal>
%<package>\NeedsTeXFormat{LaTeX2e}[2020/10/01]
%<package>\RequirePackage{expl3}
%<+package>\GetIdInfo$Id: termsim.dtx 1.1.1 2021-12-26 08:00:00 +0800 Nan Geng <nangeng@nwafu.edu.cn> $
%<package>  {Terminal simulation with tcolorbox and minted/listings.}
%<package>\ProvidesExplPackage{\ExplFileName}
%<package>  {\ExplFileDate}{\ExplFileVersion}{\ExplFileDescription}
%<*driver>
\documentclass{ctxdoc}
\usepackage{minted}
\usepackage{listings}
\usepackage{xcolor}
\usepackage{xcolor-material}
\usepackage{termsim}
\NewDocumentCommand{\init}{+v}{\hspace{\fill}初始值~=~\textcolor{blue}{\bfseries#1}}
\DeclareDocumentCommand\opt{m}{\texttt{#1}}
\DeclareDocumentCommand\kvopt{mm}
  {\texttt{#1\breakablethinspace=\breakablethinspace#2}}
\def\breakablethinspace{\hskip 0.16667em\relax}
\begin{document}
%<!--CODEDOC-->  \DisableImplementation
  \EnableImplementation
  \DocInput{\jobname.dtx}
  \IndexLayout
  \PrintChanges
  \PrintIndex
\end{document}
%</driver>
% \fi
%
% \changes{v1.0.0}{2021/12/14}{开始编写模板}
%
% \CheckSum{318}
% \GetFileId{termsim.sty}
%
% \CharacterTable
%  {Upper-case    \A\B\C\D\E\F\G\H\I\J\K\L\M\N\O\P\Q\R\S\T\U\V\W\X\Y\Z
%   Lower-case    \a\b\c\d\e\f\g\h\i\j\k\l\m\n\o\p\q\r\s\t\u\v\w\x\y\z
%   Digits        \0\1\2\3\4\5\6\7\8\9
%   Exclamation   \!     Double quote  \"     Hash (number) \#
%   Dollar        \$     Percent       \%     Ampersand     \&
%   Acute accent  \'     Left paren    \(     Right paren   \)
%   Asterisk      \*     Plus          \+     Comma         \,
%   Minus         \-     Point         \.     Solidus       \/
%   Colon         \:     Semicolon     \;     Less than     \<
%   Equals        \=     Greater than  \>     Question mark \?
%   Commercial at \@     Left bracket  \[     Backslash     \\
%   Right bracket \]     Circumflex    \^     Underscore    \_
%   Grave accent  \`     Left brace    \{     Vertical bar  \|
%   Right brace   \}     Tilde         \~}
%
%
% \title{\bfseries\pkg{termsim}:终端窗口模拟宏包\\\bfseries\pkg{termsim}:terminal simulation package}
% \author{耿楠\\ \path{nangeng@nwafu.edu.cn}}
% \date{\filedate\qquad\fileversion\thanks{\url{https://github.com/registor/termsim}}\thanks{\url{https://gitee.com/nwafu_nan/termsim}}}
% \maketitle
%
% \begin{documentation}
%
% \begin{abstract}
%
% \pkg{termsim} 是一个基于\pkg{tcolorbox}宏包,用\LaTeX3 开发的 \LaTeX 宏包,
% 它提供了用于模拟Win10、Ubuntu和Mac的终端窗口的\env{terminal}环境和\tn{termfile}命令,
% \env{terminal}环境将环境中的内容在模拟的终端窗口中输出,
% 而\tn{termfile}命令用于将指定的文件读入后在模拟的终端窗口中输出。
% 同时提供了\env{terminal*}环境和\tn{termfile*}命令,用于模拟带底线注释的终端窗口。
% 操作系统类型和终端窗口颜色主题等样式,
% 可以通过\tn{termset}命令或\env{terminal}(\env{terminal*})环境和
% \tn{termfile}(\tn{termfile*})命令的可选项进行设置。
%
% \end{abstract}
%
% \tableofcontents
%
% \section{简要说明}
%
% \pkg{termsim} 是一个用于模拟Win10、Ubuntu和Mac终端窗口的\LaTeX{}宏包,
% 用于实现对Win10、Mac或Ubuntu终端窗口的模拟排版。
%
% \section{用户接口}
%
% \subsection{\env{terminal}和\env{terminal*}环境}
%
% \begin{function}[added=2021-12-17,updated=2021-12-17]{terminal,terminal*}
%   \begin{syntax}
%     \tn{begin}\{terminal\} \oarg{外观选项} \Arg{窗口标题}
%     .....
%     \tn{end}\{terminal\}
%
%     \tn{begin}\{terminal*\} \oarg{外观选项} \oarg{底线注释} \Arg{窗口标题}
%     .....
%     \tn{end}\{terminal*\}
%   \end{syntax}
%
%   排版终端窗口,内容来自于由环境内容,
%   可以通过\Arg{窗口标题}为窗口设置标题。
%
%   \oarg{外观选项}中通过key-value方式设置终端窗口的外观。
%
%   星号环境\texttt{terminal*}用于为终端窗口添加底线注释,
%   注意此时需要在\oarg{外观选项}后提供\oarg{底线注释}内容。
%
%   终端窗口的外观也可以通过\tn{termset}命令的逗号分隔
%   key-value列表进行设置。
%
% \end{function}
%   \begin{Verbatim}[frame=none,numbers=left,gobble=5]
%     \centering
%     \termset{os=win10,theme=dark,shstretch=1.0}
%     \begin{terminal}{窗口标题}
%       C:\Users\Administrator 的目录
%       2018/07/21  15:39    <DIR>          .
%       2018/07/21  15:39    <DIR>          ..
%       2018/07/13  08:35    <DIR>          3D Objects
%       2018/07/13  08:35    <DIR>          Contacts
%       2018/08/06  07:46    <DIR>          Desktop
%                       2 个文件      6,029,312 字节
%                      18 个目录 14,035,107,840 可用字节
%     \end{terminal}
%   \end{Verbatim}
%
%\begin{center}
%\begin{minipage}{0.85\textwidth}
%\termset{os=win10,theme=dark,shstretch=1.0}
%\begin{terminal}{窗口标题}
%C:\Users\Administrator 的目录
%2018/07/21  15:39    <DIR>          .
%2018/07/21  15:39    <DIR>          ..
%2018/07/13  08:35    <DIR>          3D Objects
%2018/07/13  08:35    <DIR>          Contacts
%2018/08/06  07:46    <DIR>          Desktop
%                2 个文件      6,029,312 字节
%               18 个目录 14,035,107,840 可用字节
%\end{terminal}
%\end{minipage}
%\end{center}
%
% \subsection{\tn{termfile}和\tn{termfile*}命令}
%
% \begin{function}[added=2021-12-18,updated=2021-12-18]{\termfile,\termfile*}
%   \begin{syntax}
%     \tn{termfile}  \oarg{外观选项} \Arg{窗口标题} \Arg{内容文件}
%     \tn{termfile*} \oarg{外观选项} \oarg{底线注释} \Arg{窗口标题} \Arg{内容文件}
%   \end{syntax}
%
%   排版终端窗口,内容来自于由\Arg{内容文件},
%   可以通过\Arg{窗口标题}为窗口设置标题。
%
%   \oarg{外观选项}中通过key-value方式设置终端窗口的外观。
%
%   星号命令\tn{termfile*}用于为终端窗口添加底线注释,
%   注意此时需要在\oarg{外观选项}后提供\oarg{底线注释}内容。
%
%   排版样式也可以通过\tn{termset}命令的逗号分隔的
%   key-value列表进行设置。
% \end{function}
%
% \newpage
%
%   \begin{Verbatim}[frame=none,numbers=left,gobble=5]
%     \centering
%     \termfile{窗口标题}{test.bat}
%   \end{Verbatim}
%
%\begin{center}
%\begin{minipage}{0.85\textwidth}
%\termfile{窗口标题}{test.bat}
%\end{minipage}
%\end{center}
%
%\subsection{\tn{termset}命令}
%
% \begin{function}[added=2021-12-18,updated=2021-12-18]{\termset}
%   \begin{syntax}
%     \tn{termset} \marg{键值列表}
%   \end{syntax}
%   \tn{termset} 的参数是一组由(英文)逗号分隔的选项列表,
%   列表中的选项采用 \kvopt{\meta{key}}{\meta{value}} 形式。
%   部分选项的\meta{value} 可以省略。
%   对于同一选项,后续设置会覆盖以前的设置。
%
%   多数选项都设有默认值。
%
%   \tn{termset} 采用 \LaTeX3 风格的键值设置,支持不同类型以及多种
%   层次的选项设定。键值列表中,“|=|”左右的空格不影响设置;但需注意,
%   参数列表中\textbf{不可以出现空行}。
%
%   布尔型的参数 \kvopt{\meta{选项}}{true} 中的“|= true|”可以省略。
% \end{function}
%
% \section{选项说明}
%
%   \pkg{termsim}宏包提供了一系列选项,以设置终端窗口外观样式。
%   载入\pkg{termsim}宏包后,以下选项均可通过用户接口命令\tn{termset}进行设置。
%   同时,这些选项也可以通过\env{terminal}(\env{terminal*})环境或
%   \tn{termfile}(\tn{termfile*})命令的\oarg{外观选项}进行设置。
%
% \subsection{代码引擎}
%
% \begin{function}[added=2021-12-18,updated=2021-12-18]{minted}
%   \begin{syntax}
%     minted = <\TTF> \init{true}
%   \end{syntax}
%   \opt{minted} 用于设置代码排版引擎,如为\textbf{true}则使用\pkg{minted}宏包
%   排版终端窗口中的内容,如为\textbf{false}则使用\pkg{listings}宏包排版
%   终端窗口中的内容。默认值为\textbf{true}。
% \end{function}
%
% \subsection{操作系统}
%
% \begin{function}[added=2021-12-18,updated=2021-12-18]{os}
%   \begin{syntax}
%     os =  <win10|mac|ubuntu> \init{win10}
%   \end{syntax}
%   \opt{os} 用于设置终端窗口的操作系统类型,默认值为\textbf{win10}。
% \end{function}
%
% \subsection{颜色主题}
%
% \begin{function}[added=2021-12-18,updated=2021-12-18]{theme}
%   \begin{syntax}
%     theme =  <dark|light|white> \init{dark}
%   \end{syntax}
%   \opt{theme} 用于设置终端窗口的颜色主题,默认值为\textbf{dark}。
% \end{function}
%
% \subsection{代码样式}
%
% \begin{function}[added=2021-12-18,updated=2021-12-18]{shstyle}
%   \begin{syntax}
%     shstyle = \Arg{代码样式} \init{bw}
%   \end{syntax}
%   \opt{shstyle} 用于设置终端窗口内容(代码)的样式,仅对\textbf{minted}
%   引擎有效,默认值为\textbf{bw}。
% \end{function}
%
% \subsection{代码字号}
%
% \begin{function}[added=2021-12-18,updated=2021-12-18]{shsize}
%   \begin{syntax}
%     shsize = \Arg{字号命令} \init{\small}
%   \end{syntax}
%   \opt{shsize} 用于设置终端窗口内容(代码)的字号,仅对\textbf{minted}
%   引擎有效,默认值为\textbf{\tn{small}}。
% \end{function}
%
% \subsection{底线注释格式}
%
% \begin{function}[added=2021-12-18,updated=2021-12-18]{commentf}
%   \begin{syntax}
%     commentf = \Arg{格式命令组} \init{\small\sffamily}
%   \end{syntax}
%   \opt{commentf} 用于设置终端窗口底线注释的格式,
%   默认值为\textbf{\tn{small}\tn{sffamily}}。
% \end{function}
%
% \subsection{代码行距}
%
% \begin{function}[added=2021-12-18,updated=2021-12-18]{shstretch}
%   \begin{syntax}
%     shstretch = \Arg{浮点数} \init{1.0}
%   \end{syntax}
%   \opt{shstretch} 用于设置终端窗口内容(代码)的行距,仅对\textbf{minted}
%   引擎有效,取浮点数。默认值为\textbf{1.0}。
% \end{function}
%
% \newpage
%
% \section{排版样例}
%
% \pkg{termsim}宏包可用于需要排版操作系统终端窗口的场合,
% 以避免使用截图。终端窗口可以带/不带底线注释,也可以
% 是暗色(dark)、亮色(light)或白色(white)颜色主题。
%
% \subsection{Win10终端窗口}
%
% \subsubsection{\env{terminal}/\env{terminal*}环境}
%
%   \begin{Verbatim}[frame=none,numbers=left,gobble=5]
%     \centering
%     \termset{os=win10,theme=dark,shstretch=1.0}
%     \begin{terminal}{窗口标题}
%       C:\Users\Administrator 的目录
%       2018/07/21  15:39    <DIR>          .
%       2018/07/21  15:39    <DIR>          ..
%       2018/07/13  08:35    <DIR>          3D Objects
%       2018/07/13  08:35    <DIR>          Contacts
%       2018/08/06  07:46    <DIR>          Desktop
%                       2 个文件      6,029,312 字节
%                      18 个目录 14,035,107,840 可用字节
%     \end{terminal}
%   \end{Verbatim}
%
%\begin{center}
%\begin{minipage}{0.85\textwidth}
%\termset{os=win10,theme=dark,shstretch=1.0}
%\begin{terminal}{窗口标题}
%C:\Users\Administrator 的目录
%2018/07/21  15:39    <DIR>          .
%2018/07/21  15:39    <DIR>          ..
%2018/07/13  08:35    <DIR>          3D Objects
%2018/07/13  08:35    <DIR>          Contacts
%2018/08/06  07:46    <DIR>          Desktop
%                2 个文件      6,029,312 字节
%               18 个目录 14,035,107,840 可用字节
%\end{terminal}
%\end{minipage}
%\end{center}
%
%   \begin{Verbatim}[frame=none,numbers=left,gobble=5]
%     \centering
%     \begin{terminal*}[os=win10,theme=light][底线注释]{窗口标题}
%       C:\Users\Administrator 的目录
%       2018/07/21  15:39    <DIR>          .
%       2018/07/21  15:39    <DIR>          ..
%       2018/07/13  08:35    <DIR>          3D Objects
%       2018/07/13  08:35    <DIR>          Contacts
%       2018/08/06  07:46    <DIR>          Desktop
%                       2 个文件      6,029,312 字节
%                      18 个目录 14,035,107,840 可用字节
%     \end{terminal*}
%   \end{Verbatim}
%
%\begin{center}
%\begin{minipage}{0.85\textwidth}
%\begin{terminal*}[os=win10,theme=light][底线注释]{窗口标题}
%C:\Users\Administrator 的目录
%2018/07/21  15:39    <DIR>          .
%2018/07/21  15:39    <DIR>          ..
%2018/07/13  08:35    <DIR>          3D Objects
%2018/07/13  08:35    <DIR>          Contacts
%2018/08/06  07:46    <DIR>          Desktop
%                2 个文件      6,029,312 字节
%               18 个目录 14,035,107,840 可用字节
%\end{terminal*}
%\end{minipage}
%\end{center}
%
%
% \subsubsection{\tn{termfile}/\tn{termfile*}命令}
%
%   \begin{Verbatim}[frame=none,numbers=left]
%     \centering
%     \termset{os=win10,theme=dark}
%     \termfile{窗口标题}{test.bat}
%   \end{Verbatim}
%
%\begin{center}
%\begin{minipage}{0.85\textwidth}
%\termset{os=win10,theme=dark}
%\termfile{窗口标题}{test.bat}
%\end{minipage}
%\end{center}
%
%   \begin{Verbatim}[frame=none,numbers=left]
%     \centering
%     \termfile*[os=win10,theme=light][底线注释]{窗口标题}{test.bat}
%   \end{Verbatim}
%
%\begin{center}
%\begin{minipage}{0.85\textwidth}
%\termfile*[os=win10,theme=light][底线注释]{窗口标题}{test.bat}
%\end{minipage}
%\end{center}
%
% \newpage
% \subsection{Mac终端窗口}
%
% \subsubsection{\env{terminal}/\env{terminal*}环境}
%
%   \begin{Verbatim}[frame=none,numbers=left,gobble=5]
%     \centering
%     \termset{os=mac,theme=dark}
%     \begin{terminal}{窗口标题}
%       xxxxxx@xxxxxx-lap:~$ ls
%       Desktop     Downloads   p2         Public   Templates   Videos
%       Documents   Music       Pictures   snap    '#test#'     workspace
%       xxxxxx@xxxxxx-lap:~$
%     \end{terminal}
%   \end{Verbatim}
%
%\begin{center}
%\begin{minipage}{0.85\textwidth}
%\termset{os=mac,theme=dark}
%\begin{terminal}{窗口标题}
%xxxxxx@xxxxxx-lap:~$ ls
%Desktop     Downloads   p2         Public   Templates   Videos
%Documents   Music       Pictures   snap    '#test#'     workspace
%xxxxxx@xxxxxx-lap:~$
%\end{terminal}
%\end{minipage}
%\end{center}
%
%   \begin{Verbatim}[frame=none,numbers=left,gobble=5]
%     \centering
%     \begin{terminal*}[os=mac,theme=light][底线注释]{窗口标题}
%       xxxxxx@xxxxxx-lap:~$ ls
%       Desktop     Downloads   p2         Public   Templates   Videos
%       Documents   Music       Pictures   snap    '#test#'     workspace
%       xxxxxx@xxxxxx-lap:~$
%     \end{terminal*}
%   \end{Verbatim}
%
%\begin{center}
%\begin{minipage}{0.85\textwidth}
%\begin{terminal*}[os=mac,theme=light][底线注释]{窗口标题}
%xxxxxx@xxxxxx-lap:~$ ls
%Desktop     Downloads   p2         Public   Templates   Videos
%Documents   Music       Pictures   snap    '#test#'     workspace
%xxxxxx@xxxxxx-lap:~$
%\end{terminal*}
%\end{minipage}
%\end{center}
%
% \subsubsection{\tn{termfile}/\tn{termfile*}命令}
%
%   \begin{Verbatim}[frame=none,numbers=left]
%     \centering
%     \termset{os=mac,theme=dark}
%     \termfile{窗口标题}{testls}
%   \end{Verbatim}
%
%\begin{center}
%\begin{minipage}{0.85\textwidth}
%\termset{os=mac,theme=dark}
%\termfile{窗口标题}{testls}
%\end{minipage}
%\end{center}
% \newpage
%   \begin{Verbatim}[frame=none,numbers=left]
%     \centering
%     \termfile*[os=mac,theme=light][底线注释]{窗口标题}{testls}
%   \end{Verbatim}
%
%\begin{center}
%\begin{minipage}{0.85\textwidth}
%\termfile*[os=mac,theme=light][底线注释]{窗口标题}{testls}
%\end{minipage}
%\end{center}
%
% \subsection{Ubuntu终端窗口}
%
% \subsubsection{\env{terminal}/\env{terminal*}环境}
%
%   \begin{Verbatim}[frame=none,numbers=left,gobble=5]
%     \centering
%     \termset{os=ubuntu,theme=dark}
%     \begin{terminal}{窗口标题}
%       xxxxxx@xxxxxx-lap:~/Desktop $
%       git clone git@gitee.com:xxxxxx/xxxxxx
%       正克隆到 'xxxxxx'...
%       remote: Enumerating objects: 106, done.
%       remote: Counting objects: 100% (106/106), done.
%       remote: Compressing objects: 100% (105/105), done.
%       remote: Total 106 (delta 9), reused 0 (delta 0)
%       接收对象中: 100% (106/106), 237.88 MiB | 730.00 KiB/s, 完成.
%       处理 delta 中: 100% (9/9), 完成.
%       正在检出文件: 100% (75/75), 完成.
%       xxxxxx@xxxxxx-lap:~/Desktop $
%     \end{terminal}
%   \end{Verbatim}
%
%\begin{center}
%\begin{minipage}{0.85\textwidth}
%\termset{os=ubuntu,theme=dark}
%\begin{terminal}{窗口标题}
%xxxxxx@xxxxxx-lap:~/Desktop $
%git clone git@gitee.com:xxxxxx/xxxxxx
%正克隆到 'xxxxxx'...
%remote: Enumerating objects: 106, done.
%remote: Counting objects: 100% (106/106), done.
%remote: Compressing objects: 100% (105/105), done.
%remote: Total 106 (delta 9), reused 0 (delta 0)
%接收对象中: 100% (106/106), 237.88 MiB | 730.00 KiB/s, 完成.
%处理 delta 中: 100% (9/9), 完成.
%正在检出文件: 100% (75/75), 完成.
%xxxxxx@xxxxxx-lap:~/Desktop $
%\end{terminal}
%\end{minipage}
%\end{center}
%
%   \begin{Verbatim}[frame=none,numbers=left,gobble=5]
%     \centering
%     \begin{terminal*}[os=ubuntu,theme=light][底线注释]{窗口标题}
%       xxxxxx@xxxxxx-lap:~/Desktop $
%       git clone git@gitee.com:xxxxxx/xxxxxx
%       正克隆到 'xxxxxx'...
%       remote: Enumerating objects: 106, done.
%       remote: Counting objects: 100% (106/106), done.
%       remote: Compressing objects: 100% (105/105), done.
%       remote: Total 106 (delta 9), reused 0 (delta 0)
%       接收对象中: 100% (106/106), 237.88 MiB | 730.00 KiB/s, 完成.
%       处理 delta 中: 100% (9/9), 完成.
%       正在检出文件: 100% (75/75), 完成.
%       xxxxxx@xxxxxx-lap:~/Desktop $
%     \end{terminal*}
%   \end{Verbatim}
%
%\newpage
%\begin{center}
%\begin{minipage}{0.85\textwidth}
%\begin{terminal*}[os=ubuntu,theme=light][底线注释]{窗口标题}
%xxxxxx@xxxxxx-lap:~/Desktop $
%git clone git@gitee.com:xxxxxx/xxxxxx
%正克隆到 'xxxxxx'...
%remote: Enumerating objects: 106, done.
%remote: Counting objects: 100% (106/106), done.
%remote: Compressing objects: 100% (105/105), done.
%remote: Total 106 (delta 9), reused 0 (delta 0)
%接收对象中: 100% (106/106), 237.88 MiB | 730.00 KiB/s, 完成.
%处理 delta 中: 100% (9/9), 完成.
%正在检出文件: 100% (75/75), 完成.
%xxxxxx@xxxxxx-lap:~/Desktop $
%\end{terminal*}
%\end{minipage}
%\end{center}
%
% \subsubsection{\tn{termfile}/\tn{termfile*}命令}
%
%   \begin{Verbatim}[frame=none,numbers=left]
%     \centering
%     \termset{os=ubuntu,theme=dark}
%     \termfile{窗口标题}{testsh}
%   \end{Verbatim}
%
%\begin{center}
%\begin{minipage}{0.85\textwidth}
%\termset{os=ubuntu,theme=dark}
%\termfile{窗口标题}{testsh}
%\end{minipage}
%\end{center}
%
%   \begin{Verbatim}[frame=none,numbers=left]
%     \centering
%     \termfile*[os=ubuntu,theme=light][底线注释]{窗口标题}{testsh}
%   \end{Verbatim}
%
%\begin{center}
%\begin{minipage}{0.85\textwidth}
%\termfile*[os=ubuntu,theme=light][底线注释]{窗口标题}{testsh}
%\end{minipage}
%\end{center}
%
%  也可以使用\texttt{white}颜色主题,使用白底黑字排版,以便
%  打印输出。
%
%   \begin{Verbatim}[frame=none,numbers=left,gobble=5]
%     \centering
%     \termset{os=ubuntu,theme=white}
%     \begin{terminal}{窗口标题}
%       xxxxxx@xxxxxx-lap:~/Desktop $
%       git clone git@gitee.com:xxxxxx/xxxxxx
%       正克隆到 'xxxxxx'...
%       remote: Enumerating objects: 106, done.
%       remote: Counting objects: 100% (106/106), done.
%       remote: Compressing objects: 100% (105/105), done.
%       remote: Total 106 (delta 9), reused 0 (delta 0)
%       接收对象中: 100% (106/106), 237.88 MiB | 730.00 KiB/s, 完成.
%       处理 delta 中: 100% (9/9), 完成.
%       正在检出文件: 100% (75/75), 完成.
%       xxxxxx@xxxxxx-lap:~/Desktop $
%     \end{terminal}
%   \end{Verbatim}
%
%\begin{center}
%\begin{minipage}{0.85\textwidth}
%\termset{os=ubuntu,theme=white}
%\begin{terminal}{窗口标题}
%xxxxxx@xxxxxx-lap:~/Desktop $
%git clone git@gitee.com:xxxxxx/xxxxxx
%正克隆到 'xxxxxx'...
%remote: Enumerating objects: 106, done.
%remote: Counting objects: 100% (106/106), done.
%remote: Compressing objects: 100% (105/105), done.
%remote: Total 106 (delta 9), reused 0 (delta 0)
%接收对象中: 100% (106/106), 237.88 MiB | 730.00 KiB/s, 完成.
%处理 delta 中: 100% (9/9), 完成.
%正在检出文件: 100% (75/75), 完成.
%xxxxxx@xxxxxx-lap:~/Desktop $
%\end{terminal}
%\end{minipage}
%\end{center}
%
% \end{documentation}
%
% \StopEventually{}
%
% \begin{implementation}
%
% \section{代码实现}
%
% 该宏包使用 \LaTeX3 语法编写,依赖 \pkg{expl3} 环境,
% 并需调用 \pkg{l3packages}、\pkg{tcolorbox}、\pkg{minted}
% \pkg{listings}、\pkg{fontawesome5}等宏包。
%
% 按照 \LaTeX3 语法,代码中的空格、换行、回车与制表符会完全被忽略,
% 而下划线“|_|”和冒号“|:|”则可作为一般字母使用。
% 正常的空格可以使用“|~|”代替;至于 |~| 原来所表示的“带子”,
% 则要用 \LaTeXe{} 的原始命令 \tn{nobreakspace} 代替。
%
% 以下代码中有一些形如 \textcolor[HTML]{2E3191}{\textsf{<*package>}}
% 的标记,这是 \pkg{DocStrip} 中的“guard”,用来选择性地提取文件。
% “\textsf{*}”和“\textsf{/}”分别表示该部分的开始和结束。不含
% “\textsf{*}”和“\textsf{/}”的 guard 出现在行号右侧,它们用来确定
% 单独一行代码的归属。这些 guard 的颜色深浅不一,用以明确嵌套关系。
%
% 另有若干形如 \textcolor[HTML]{AD1457}{\textsf{<@@=termsim>}} 的 guard,
% 它们由 \pkg{l3docstrip} 定义,用来标识名字空间(模块)。
%
% \subsection{环境检测与准备}
%
%    \begin{macrocode}
%<*package>
%<@@=termsim>
%    \end{macrocode}
%
% \subsection{载入宏包}
%
% 载入必要的宏包
%
%    \begin{macrocode}
\RequirePackage { xtemplate, l3keys2e, xparse }
%    \end{macrocode}
%
% 载入\pkg{fontawesome5}、\pkg{tcolorbox}、\pkg{varwidth}、
% \pkg{amssymb}、\pkg{xcolor}、\pkg{etoolbox}宏包。
%
%    \begin{macrocode}
\RequirePackage {fontawesome5, tcolorbox, varwidth, amssymb, xcolor, etoolbox}
%    \end{macrocode}
%
% 解决Expl3中\pkg{tcolorbox}和\pkg{tikz}附加库无法加载的
% 补丁(需要\pkg{etoolbox}宏包支持)
%
% \url{https://tex.stackexchange.com/questions/57424/using-of-usetikzlibrary-in-an-expl3-package-fails}
%
%    \begin{macrocode}
\ExplSyntaxOff
\patchcmd
{\tcb@input@library@in}
  {%
   \input\tcbpkgprefix#1\relax%
  }
  {%
    \@pushfilename
    \input\tcbpkgprefix#1\relax%
    \@popfilename
  }
  {}{}

\patchcmd
{\pgfutil@InputIfFileExists}
  {\input #1}
  {%
    \@pushfilename
    \xdef\@currname{#1}%
    \input #1 %
    \@popfilename
  }
  {}{}
\ExplSyntaxOn
%    \end{macrocode}
%
% 载入\pkg{tcolorbox}和\pkg{tikz}附加库
%
%    \begin{macrocode}
\tcbuselibrary{skins, xparse, breakable}
\usetikzlibrary{shapes.geometric}
%    \end{macrocode}
%
% \subsection{定义颜色}
%
%    \begin{macrocode}
\definecolor{mpurple}{RGB}{48,10,36}
\definecolor{mgray}{RGB}{70,72,67}
\definecolor{ogray}{RGB}{148,147,141}
\definecolor{oorange}{RGB}{233,101,56}
\definecolor{termimal}{RGB}{80,78,70}
\definecolor{linux}{RGB}{0,39,51}
\definecolor{windows}{HTML}{00B294}
\definecolor{cvgrayc}{RGB}{247,247,247}
\definecolor{cvgray}{RGB}{220,220,220}
\definecolor{cvgrayb}{RGB}{153,153,153}
\definecolor{cvblue}{RGB}{223,238,255}
\definecolor{zhanqing}{RGB}{0,51,113}
\definecolor{chengse}{RGB}{250,140,53}

\definecolor{AppleRed}{RGB}{255,95,86}
\definecolor{AppleYellow}{RGB}{255,189,46}
\definecolor{AppleGreen}{RGB}{39,201,63}
\definecolor{AppleGray}{HTML}{D8D6D9}

\definecolor{WinGray}{HTML}{FFFFFF}
\definecolor{WinBlue}{HTML}{1883D7}
%    \end{macrocode}
%
% \subsection{终端盒子图标设计}
%
% 用TikZ绘制Ubuntu终端盒子的最小化、关闭和最大化图标
%
%    \begin{macrocode}
\newcommand{\UbuntuMin}{%
  \begin{tikzpicture}[x=2.4ex,y=2.4ex,line~width=0.15ex,scale=1]
    \shade[shading=ball,left~color=ogray,right~color=ogray!50!white] (0,0) circle (0.5);
    \draw[termimal](-0.3,0)--(0.3,0);
  \end{tikzpicture}
}
\newcommand{\UbuntuClose}{%
  \begin{tikzpicture}[x=2.4ex,y=2.4ex,line~width=0.15ex,scale=1]
    \shade[shading=ball,left~color=oorange,right~color=oorange!70!white] (0,0) circle (0.5);
    \draw[termimal](-0.25,-0.25)--(0.25,0.25);
    \draw[termimal](-0.25,0.25)--(0.25,-0.25);
  \end{tikzpicture}
}
\newcommand{\UbuntuMax}{%
  \begin{tikzpicture}[x=2.4ex,y=2.4ex,line~width=0.15ex,scale=1]
    \shade [shading=ball,left~color=ogray,right~color=ogray!50!white] (0,0) circle (0.5);
    \draw[termimal](-0.25,-0.2)rectangle(0.25,0.2);
  \end{tikzpicture}
}
%    \end{macrocode}
%
% 绘制Windows Logo
%
%    \begin{macrocode}
\newcommand{\WindowsLogo}{%
  \begin{tikzpicture}[square/.style={regular~polygon,regular~polygon~sides=4},
         scale=0.65, every~node/.style={scale=0.65}]
    \node[square,draw,fill=black,text=white] at (0, 0) (a) {\small\faTerminal};
  \end{tikzpicture}
}
%    \end{macrocode}
%
% \subsection{终端盒子样式设计}
%
% 定义Mac终端盒子的tcolorbox样式
%
%    \begin{macrocode}
\tcbset{%
  skin=enhanced,%
  apple/.style={%
    halign~title=center,%
    skin=bicolor,%
    boxrule=0.5mm,%
    breakable,%
    frame~style={%
      draw=AppleGray,%
      left~color=AppleGray,%
      right~color=AppleGray%
    },%
    overlay~unbroken = {%
      \node[inner~sep=0pt,anchor=north~west,yshift=-4pt,xshift=10pt,text=white]
        at (frame.north~west) {\textcolor{AppleRed}{\faCircle}$\,$
                               \textcolor{AppleYellow}{\faCircle}$\,$
                               \textcolor{AppleGreen}{\faCircle}};
    },%
    overlay~first = {%
      \node[inner~sep=0pt,anchor=north~west,yshift=-4pt,xshift=10pt,text=white]
        at (frame.north~west) {\textcolor{AppleRed}{\faCircle}$\,$
                               \textcolor{AppleYellow}{\faCircle}$\,$
                               \textcolor{AppleGreen}{\faCircle}};
    },%
    fonttitle=\sffamily,%\bfseries,%
    fontupper=\small\sffamily,%
    fontlower=\small\sffamily%
  },
  appledark/.style={%
    apple,%
    colback=black,%
    colupper=white,%
    colbacktitle=AppleGray,%
    colframe=AppleGray,%
    coltitle=black%
    %coltext=MyLightGray
  },
  applelight/.style={%
    apple,%
    colback=cvgrayc,%
    colupper=black,%
    colbacktitle=AppleGray,%
    colframe=AppleGray,%
    coltitle=black%
  },%
  applewhite/.style={%
    apple,%
    colback=white,%
    colupper=black,%
    colbacktitle=AppleGray,%
    colframe=AppleGray,%
    coltitle=black%
  },%
}%
%    \end{macrocode}
%
% \changes{v1.1.0}{2021/12/24}{调整Win10 Logo的yshift}
%
% 定义Win10终端盒子的tcolorbox样式
%
%    \begin{macrocode}
\tcbset{%
  skin=enhanced,%
  win10/.style={%
    skin=bicolor,%
    boxrule=0.1mm,%
    %toptitle=1ex,
    sharp~corners,
    breakable,%
    colbacktitle=WinGray,%
    colframe=WinGray,%
    coltitle=black,%
    fonttitle=\sffamily,%\bfseries,
    fontupper=\small\sffamily,
    fontlower=\small\sffamily,
    frame~style={%
      draw=WinBlue,%
      left~color=WinBlue,%
      right~color=WinBlue%
    },%
    overlay~unbroken = {%
      \node[inner~sep=0pt,anchor=north~west,yshift=-2pt,xshift=1.2pt,text=black]
        at (frame.north~west){\WindowsLogo};
      \node[inner~sep=0pt,anchor=north~east,yshift=-4pt,xshift=-8pt,text=black]
        at (frame.north~east){\rule{0.8em}{0.6pt}\quad$\square$\quad{\Large$\times$}};
    },%
    overlay~first = {%
      \node[inner~sep=0pt,anchor=north~west,yshift=-2pt,xshift=1.0pt,text=black]
        at (frame.north~west){\WindowsLogo};
      \node[inner~sep=0pt,anchor=north~east,yshift=-4pt,xshift=-8pt,text=black]
        at (frame.north~east){\rule{0.8em}{0.6pt}\quad$\square$\quad{\Large$\times$}};
    }%
  },
  win10dark/.style={
    win10,%
    colback=black,%
    colupper=white,%
    %coltext=white%
  },
  win10light/.style={
    win10,%
    colback=white,%
    colupper=black,%
    %coltext=black%
  },
  win10white/.style={
    win10,%
    colback=white,%
    colupper=black,%
    %coltext=white%
  },
}%
%    \end{macrocode}
%
% 定义Ubuntu终端盒子的tcolorbox样式
%
%    \begin{macrocode}
\tcbset{%
  skin=enhanced,%
  ubuntu/.style={%
    halign~title=center,%
    skin=bicolor,%
    boxrule=0.5mm,%
    breakable,% 
    overlay~unbroken = {%
      \node[inner~sep=0pt,anchor=north~east,yshift=-3.5pt,xshift=-5pt,text=white]
        at (frame.north~east){\UbuntuMin$\,$\UbuntuMax$\,$\UbuntuClose};%
    },%
    overlay~first = {%
      \node[inner~sep=0pt,anchor=north~east,yshift=-3.5pt,xshift=-5pt,text=white]
        at (frame.north~east){\UbuntuMin$\,$\UbuntuMax$\,$\UbuntuClose};%
    },%
    fonttitle=\sffamily,%
    fontupper=\small\sffamily,%
    fontlower=\small\sffamily%
  },
  udark/.style={%
    ubuntu,%
    frame~style={%
      draw=mgray,%
      left~color=mgray,%
      right~color=mgray%
    },%
    colback=mpurple,%
    colupper=white,%
    colbacktitle=mgray,%
    colframe=mgray,%
    coltitle=white%
    %coltext=MyLightGray
  },
  ulight/.style={%
    ubuntu,%
    frame~style={%
      draw=cvgrayb,%
      left~color=cvgrayb,%
      right~color=cvgrayb%
    },%
    colback=cvgrayc,%
    colupper=black,%
    colbacktitle=cvgrayb,%
    colframe=cvgrayb,%
    coltitle=black%
  },%
  uwhite/.style={%
    ubuntu,%
    frame~style={%
      draw=cvgrayb,%
      left~color=cvgrayb,%
      right~color=cvgrayb%
    },%
    colback=white,%
    colupper=black,%
    colbacktitle=cvgrayb,%
    colframe=cvgrayb,%
    coltitle=black%
  },%
}%
%    \end{macrocode}
%
% \subsection{内部变量声明}
%
% \begin{variable}{\l_@@_minted_bool,
% \l_@@_comment_bool,
% \l_@@_os_type_int,
% \l_@@_color_theme_int,
% \l_@@_code_style_tl,
% \l_@@_code_fontsize_tl,
% \l_@@_comment_format_tl,
% \l_@@_shell_baseline_stretch_fp
%  }
% 定义变量。
%
%    \begin{macrocode}
\bool_new:N   \l_@@_minted_bool
\bool_new:N   \l_@@_comment_bool
\int_new:N    \l_@@_os_type_int
\int_new:N    \l_@@_color_theme_int
%
\tl_new:N     \l_@@_shell_style_tl
\tl_new:N     \l_@@_shell_fontsize_tl
\tl_new:N     \l_@@_comment_format_tl
\fp_new:N     \l_@@_shell_baseline_stretch_fp
%    \end{macrocode}
% \end{variable}
%
% \subsection{选项处理}
%
% 定义 |termsim| 键值类。
%    \begin{macrocode}
\keys_define:nn { termsim }
  {
%    \end{macrocode}
%
% \begin{macro}{minted}
% 是否使用minted宏包排版代码,默认为true。
%    \begin{macrocode}
    minted .bool_set:N = \l_@@_minted_bool,
    minted .default:n = true,
    minted .initial:n = true,
%    \end{macrocode}
% \end{macro}
%
% \begin{macro}{os}
% 选择操作系统,默认为Win10。
%    \begin{macrocode}
    os .choice:,
    os .value_required:n = true,
    os .choices:nn =
      { win10, mac, ubuntu }
      { \int_set_eq:NN \l_@@_os_type_int \l_keys_choice_int },
    os .initial:n = win10,
%    \end{macrocode}
% \end{macro}
%
% \begin{macro}{theme}
% 选择终端窗口颜色主题,默认为dark。
%    \begin{macrocode}
    theme .choice:,
    theme .value_required:n = true,
    theme .choices:nn =
      { dark, light, white }
      { \int_set_eq:NN \l_@@_color_theme_int \l_keys_choice_int },
    theme .initial:n = dark,
%    \end{macrocode}
% \end{macro}
%
% \begin{macro}{shstyle}
% 代码样式
%    \begin{macrocode}
    shstyle .tl_set:N = \l_@@_shell_style_tl,
    shstyle .initial:n = bw,
%    \end{macrocode}
% \end{macro}
%
% \begin{macro}{shsize}
% 代码字号
%    \begin{macrocode}
    shsize .tl_set:N = \l_@@_shell_fontsize_tl,
    shsize .initial:n = \small,
%    \end{macrocode}
% \end{macro}
%
% \begin{macro}{commentf}
% 注释格式
%    \begin{macrocode}
    commentf .tl_set:N = \l_@@_comment_format_tl,
    commentf .initial:n = \small\sffamily,
%    \end{macrocode}
% \end{macro}
%
% \begin{macro}{shstretch}
% 基线伸展系数
%    \begin{macrocode}
    shstretch .fp_set:N  = \l_@@_shell_baseline_stretch_fp,
    shstretch .initial:n = 1.0,
%    \end{macrocode}
% \end{macro}
%
% 处理未知选项。
%    \begin{macrocode}
    unknown .code:n = { \@@_error:n { unknown-option } }
  }
\msg_new:nnn { termsim } { unknown-option }
  { package~ option~ "\l_keys_key_tl"~ is~ unknown. }
%    \end{macrocode}
%
% 参数默认值
%    \begin{macrocode}
\keys_set:nn { termsim }
  {
    minted = true,
    os     = win10,
    theme  = dark,
  }
%    \end{macrocode}
%
% \subsection{选项用户接口}
%
% \begin{macro}{\termset}
% 选项设置用户接口。
%    \begin{macrocode}
\NewDocumentCommand \termset { m }
  { \keys_set:nn { termsim } {#1} }
%    \end{macrocode}
% \end{macro}
%
% \subsection{内部函数}
%
% \changes{v1.1.0}{2021/12/24}{为minted宏包添加autogobble参数}
%
% \begin{macro}{\@@_minted_tcbset:}
% 设置终端盒子minted宏包参数。
%
%    \begin{macrocode}
\cs_new:Npn \@@_minted_tcbset:n #1
  {
    \tcbset{
      listing~engine=minted,%
      minted~style=#1,%
      minted~options={%
        autogobble,
        breaklines,%
        fontsize=\tl_use:N \l_@@_shell_fontsize_tl,%
        baselinestretch=\fp_eval:n { \l_@@_shell_baseline_stretch_fp },%
        breaksymbolleft={},%
        linenos=false,%
      },%
      minted~language=bash%
    }
  }
\cs_generate_variant:Nn \@@_minted_tcbset:n {V}
%    \end{macrocode}
% \end{macro}
%
% \begin{macro}{\@@_listings_tcbset:}
% 设置终端盒子listings宏包参数。
%
%    \begin{macrocode}
\cs_new:Npn \@@_listings_tcbset:
  {
    \lstset{% general command to set parameter(s)
      basicstyle=\ttfamily\tl_use:N \l_@@_shell_fontsize_tl, % print whole listing small
      keywordstyle=\color{chengse}\bfseries,%
      identifierstyle=, % nothing happens
      breaklines=true,
      commentstyle=\itshape\color{gray},
      stringstyle=\ttfamily, % typewriter type for strings
      numbers=left,%
      numberstyle=\tl_use:N \l_@@_shell_fontsize_tl\sffamily\color[rgb]{0.5,0.5,1.0},%
    }%
  }
\cs_generate_variant:Nn \@@_minted_tcbset:n {V}
%    \end{macrocode}
% \end{macro}
%
% \begin{macro}{\@@_comment_tcbset:}
% 设置终端盒子底线注释。
%
%    \begin{macrocode}
\cs_new:Npn \@@_comment_tcbset:n #1
  {
    \bool_if:NTF { \l_@@_comment_bool }
    {
      \tcbset{%
        listing~and~comment,%
        colbacklower=tcbcolback!5!yellow!10!white,%
        collower=linux,%
        comment={\tl_use:N \l_@@_comment_format_tl~#1},%
      }
    }{
      \tcbset{ listing~only,}
    }
  }
%    \end{macrocode}
% \end{macro}
%
% \begin{macro}{\@@_terminal_theme_select:}
% 选择终端盒子颜色主题。
%
%    \begin{macrocode}
\cs_new:Npn \@@_terminal_theme_select:
  {
    \int_case:nn {\l_@@_os_type_int}
      {
        {1}{
          \int_case:nn {\l_@@_color_theme_int}
            {
              {1}{
                \tcbset{ win10dark, }
              }
              {2}{
                \tcbset{ win10light, }
              }
              {3}{
                \tcbset{ win10white, }
              }
            }
        }
        {2}{
          \int_case:nn {\l_@@_color_theme_int}
            {
              {1}{
                \tcbset{ appledark, }
              }
              {2}{
                \tcbset{ applelight, }
              }
              {3}{
                \tcbset{ applewhite, }
              }
            }
        }
        {3}{
          \int_case:nn {\l_@@_color_theme_int}
            {
              {1}{
                \tcbset{ udark, }
              }
              {2}{
                \tcbset{ ulight, }
              }
              {3}{
                \tcbset{ uwhite, }
              }
            }
        }
    }
  }
%    \end{macrocode}
% \end{macro}
%
% \subsection{定义排版环境和命令}
%
%    \begin{macrocode}
\bool_if:NTF \l_@@_minted_bool
  {
%    \end{macrocode}
%
%  使用\pkg{minted}宏包排版代码
%
%    \begin{macrocode}
    \tcbuselibrary{minted}
%    \end{macrocode}
%
%  定义不带底线注释的窗口模拟环境\env{terminal}
%
%    \begin{macrocode}
    \NewDocumentEnvironment{ terminal }{O{} O{} m }
      {
        \bool_set_false:N  \l_@@_comment_bool
        \group_begin:
        \IfNoValueF{#1}{ \keys_set:nn { termsim } { #1 } }
        \@@_comment_tcbset:n { #2 }
        \@@_terminal_theme_select:
        \@@_minted_tcbset:V \l_@@_shell_style_tl
        \tcbset{ title={#3} }%
        \tcblisting{}
      }{
        \endtcblisting
        \group_end:
      }
%    \end{macrocode}
%
%  定义带底线注释的窗口模拟环境\env{terminal*}
%
%    \begin{macrocode}
    \NewDocumentEnvironment{ terminal* }{O{} O{} m }
      {
        \bool_set_true:N \l_@@_comment_bool
        \group_begin:
        \IfNoValueF{#1}{ \keys_set:nn { termsim } { #1 } }
        \@@_comment_tcbset:n { #2 }
        \@@_terminal_theme_select:
        \@@_minted_tcbset:V \l_@@_shell_style_tl
        \tcbset{ title={#3} }%
        \tcblisting{}
      }{
        \endtcblisting
        \group_end:
      }
%    \end{macrocode}
%
%  定义带/不带底线注释的从文件读取内容的命令\tn{termfile}/\tn{termfile*}
%
%    \begin{macrocode}
    \NewDocumentCommand \termfile { s O{} O{} m m }
      {
        \IfBooleanTF{#1}
          {
            \bool_set_true:N  \l_@@_comment_bool
          }{
            \bool_set_false:N \l_@@_comment_bool
          }

        \group_begin:
        \IfNoValueF{#2}{ \keys_set:nn { termsim } { #2 } }
        \@@_comment_tcbset:n { #3 }
        \@@_terminal_theme_select:
        \@@_minted_tcbset:V \l_@@_shell_style_tl
        \tcbset{
          title={#4},
          listing~file={#5},
        }%
        \tcbinputlisting{  }
        \group_end:
      }
  }{
%    \end{macrocode}
%
%  使用\pkg{listings}宏包排版代码,并设置代码样式。
%
%    \begin{macrocode}
    \tcbuselibrary{listings}% 用listings排版代码

    \@@_listings_tcbset:

%    \end{macrocode}
%
%  定义不带底线注释的窗口模拟环境\env{terminal}
%
%    \begin{macrocode}
    \NewDocumentEnvironment{ terminal }{O{} O{} m }
      {
        \bool_set_false:N \l_@@_comment_bool
        \group_begin:
        \IfNoValueF{#1}{ \keys_set:nn { termsim } { #1 } }
        \@@_comment_tcbset:n { #2 }
        \@@_terminal_theme_select:
        \tcbset{
          title={#3},
          listing~options={
            language=bash,
            numbers=none,
          }
        }%
        \tcblisting{}
      }{
        \endtcblisting
        \group_end:
      }
%    \end{macrocode}
%
%  定义带底线注释的窗口模拟环境\env{terminal*}
%
%    \begin{macrocode}
    \NewDocumentEnvironment{ terminal* }{O{} O{} m }
      {
        \bool_set_true:N  \l_@@_comment_bool
        \group_begin:
        \IfNoValueF{#1}{ \keys_set:nn { termsim } { #1 } }
        \@@_comment_tcbset:n { #2 }
        \@@_terminal_theme_select:
        \tcbset{
          title={#3},
          listing~options={
            language=bash,
            numbers=none,
          }
        }%
        \tcblisting{}
      }{
        \endtcblisting
        \group_end:
      }
%    \end{macrocode}
%
%  定义带/不带底线注释的从文件读取内容的命令\tn{termfile}/\tn{termfile*}
%
%    \begin{macrocode}
    \NewDocumentCommand \termfile { s O{} O{} m m }
      {
        \IfBooleanTF{#1}
          {
            \bool_set_true:N  \l_@@_comment_bool
          }{
            \bool_set_false:N \l_@@_comment_bool
          }
        \group_begin:
        \IfNoValueF{#2}{ \keys_set:nn { termsim } { #2 } }
        \@@_comment_tcbset:n { #3 }
        \@@_terminal_theme_select:
        \tcbset{
          title={#4},
          listing~file={#5},
          listing~options={
            language=bash,
            numbers=none,
          },%
        }%
        \tcbinputlisting{  }
        \group_end:
      }
  }
%    \end{macrocode}
%
%    \begin{macrocode}
\endinput
%    \end{macrocode}
%
%    \begin{macrocode}
%</package>
%    \end{macrocode}
%
% \end{implementation}
%
% \Finale
%
\endinput
